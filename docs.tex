\documentclass{article}
\usepackage[english]{babel}
\usepackage{hyperref}
\usepackage{rotating}

\begin{document}

\title{Attendbot 9000 Usage and Administration}

\author{Saul Statman}

\maketitle

\section{Usage}

\subsection{Input}

\

Input is given to the program via sending direct messages to a Slack bot account. These messages must be in the form of a ``slash command", which is a special system that Slack uses to handle messages beginning with a slash character (``/"). The bot also makes use of Slack usernames, which are the identifier used for ``@" tagging (of course, excepting the ``@").

\subsection{Date and Time Format}

\

The format used is \texttt{[MM/DD/YYYY] HH[:MM][am/pm] HH[:MM][pm/am]}. In the following section, a ``word'' is part of the format separated by strings. Any content enclosed in brackets is optional - it can be omitted and the program will fall back to a default value.

The first word is \texttt{[MM/DD/YYYY]}, which is a date formatted as MONTH-DAY-YEAR with forward slash separators. When it is omitted, it falls back to the current date.

The second and third words are \texttt{HH[:MM][am/pm]}. The only mandatory portion is the hour portion, which is a number from ``1'' to ``12''. The next section is the colon and a number from ``00'' to ``59'' representing minutes, defaulting to ``00'' when omitted. The final section is a literal ``am'' or ``pm'', representing the ante meridiem or post meridiem clarification needed for 12-hour time. The final section defaults to ``am'' in the first word and ``pm'' in the second.

The second word represents a ``begin time" and the third word represents an ``end time".

\subsection{The \texttt{/attend} Command}

\

\texttt{/attend} is the primary mode of interaction for users. The command takes arguments using the detailed date and time format. The program will read these arguments and edit the table provided. To do so, it takes in your Slack username, the date in your message, and the amount of hours between the begin and end times in your message (subtracting one hour if you were present for the lunch hour, using decimals for partial hours). The program will then add the amount of hours at the cell indexed by your username and the date.

\section{Deployment}

\subsection{Creating the Bot}

\

Firstly, you must go to \href{https://api.slack.com/apps}{Slack's apps panel} and log in. Create a new app and apply the following settings:
\begin{itemize}
	\item Basic Information $\to$ Add features and functionality $\to$ Enable only Slash Commands, Bots, Permissions
	\item Basic Information $\to$ App-Level Tokens $\to$ Generate Token and Scopes $\to$ Create a token with \texttt{connections:write}
	\item Socket Mode $\to$ Enable Socket Mode
	\item App Home $\to$ Your App’s Presence in Slack $\to$ Always Show My Bot as Online
	\item Slash Commands $\to$ Create New Command $\to$ Create the commands \texttt{/attend}, \texttt{/meeting}, and \texttt{/set}
\end{itemize}

Alternatively, go to App Manifest and paste in the JSON text stored in this repository's \texttt{manifest.json} file.

\subsection{The Spreadsheet}

\

The bot is designed to interface with spreadsheets in the following format. Note the names being on column C onward, and the dates being row 4 onward.

\begin{center} \begin{tabular}{|c||c|c|c|c} \hline
	& \begin{turn}{90}Total Meeting Hours\end{turn} & \begin{turn}{90}Slack username\end{turn} & \begin{turn}{90}Slack username\end{turn} & . . . \\ \hline
	Percent & & & & . . .\\ \hline
	Last 10\% & & & & . . .\\ \hline \hline
	& & & & . . .\\ \hline
	. . .&. . .&. . .&. . .&
\end{tabular} \end{center}

This can be modified by changing every instance of ``4" and ``C" in \texttt{main.py}'s usage of table ranges.

\subsection{The \texttt{keys} File}

\

The \texttt{keys} file is a set of keys and values in the below format, requiring the below keys. Values are replaced with placeholders.
\begin{verbatim}
APP_TOKEN=xapp-...
BOT_TOKEN=xoxb-...
SIGNING_SECRET=...
SHEET=   .   .   .
TABLE=   .   .   .
ADMINS=ID ID ID ID
\end{verbatim}

The values can be found in the following ways:
\begin{itemize}
	\item \texttt{APP\_TOKEN}: Slack apps dashboard $\to$ Basic Information $\to$ App-Level Tokens $\to$ Token
	\item \texttt{BOT\_TOKEN}: Slack apps dashboard $\to$ OAuth \& Permissions $\to$ Bot User OAuth Token
	\item \texttt{SIGNING\_SECRET}: Slack apps dashboard $\to$ Basic Information $\to$ App Credentials $\to$ Signing Secret
	\item \texttt{SHEET}: Google Sheets spreadsheet $\to$ Section in URL between \texttt{spreadsheets/d/} and \texttt{/edit}
	\item \texttt{TABLE}: Google Sheets spreadsheet $\to$ Text on the highlighted tab along the bottom of the website
	\item \texttt{ADMINS}: Space-separated list of the Slack usernames of people who have permission to administrate
\end{itemize}

\subsection{The \texttt{credentials.json} File and Google Authentication}

\

Go through the ``Set up your environment" section of \href{https://developers.google.com/sheets/api/quickstart/python#set-up-environment}{Google's Python quickstart document}. Put the resultant \texttt{credentials.json} file in the same directory as the Python files. When you first run the program (or whenever you lack a \texttt{manifest.json}), you will be prompted by Google's authentication flow. Open the stated URL and log in with a Google account that has access to the spreadsheet.

\section{Administration}

/

Only Slack accounts with their usernames listed in \texttt{ADMINS} may use administration commands.

\subsection{The \texttt{/meeting} Command}

\

For user input, the \texttt{/meeting} command is identical to the \texttt{/attend} command. If the date isn't in the first column, add it. Otherwise, select the row that it is in. Add the hours to the second column (Total Meeting Hours).

\subsection{The \texttt{/set sheet} and \texttt{/set table} Commands}

\

Take one argument after the sub-command (``sheet" or ``table"). Change the \texttt{SHEET} and \texttt{TABLE} values of the settings to that argument.

\subsection{The \texttt{/set op} and \texttt{/set deop} Commands}

\

Take one slack username as an argument after the sub-command (``op" or ``deop"). Add someone to or remove someone from the list in the \texttt{ADMINS} setting value.

\end{document}
