\documentclass{article}
\usepackage[english]{babel}
\usepackage{hyperref}
\usepackage{rotating}

\begin{document}

\title{Attendbot 9000 Usage and Administration}

\author{Saul Statman}

\maketitle

\section{Usage}

\subsection{Input}

\

Input is given to the program via running shortcuts provided by a Slack bot account. These shortcuts may be acessed from the ``[/]" menu on the message composition bar. The bot also makes use of Slack usernames, which are the identifier used for ``@" tagging (of course, excepting the ``@").

\subsection{The \texttt{Attend} Modal}

\

\texttt{Attend} is the primary mode of interaction for users. The shortcut will show a modal which takes input for the date, start time, and end time. The program will read these arguments and edit the table provided. To do so, it takes in your Slack username, the date in your message, and the amount of hours between the begin and end times in your message (subtracting one hour if you were present for the lunch hour, using decimals for partial hours). The program will then add the amount of hours at the cell indexed by your username and the date.

\section{Deployment}

\subsection{Creating the Bot}

\

Firstly, you must go to \href{https://api.slack.com/apps}{Slack's apps panel} and log in. Create a new app and go to App Manifest. Paste in the JSON text stored in this repository's \texttt{manifest.json} file.

\subsection{The Spreadsheet}

\

The bot is designed to interface with spreadsheets in the following format. Note the names being on column C onward, and the dates being row 4 onward.

\begin{center} \begin{tabular}{|c||c|c|c|c} \hline
	& \begin{turn}{90}Total Meeting Hours\end{turn} & \begin{turn}{90}Slack username\end{turn} & \begin{turn}{90}Slack username\end{turn} & . . . \\ \hline
	Percent & & & & . . .\\ \hline
	Last 10\% & & & & . . .\\ \hline \hline
	& & & & . . .\\ \hline
	. . .&. . .&. . .&. . .&
\end{tabular} \end{center}

This can be modified by changing every instance of ``4" and ``C" in \texttt{main.py}'s usage of table ranges.

\subsection{The \texttt{keys} File}

\

The \texttt{keys} file is a set of keys and values in the below format, requiring the below keys. Values are replaced with placeholders.
\begin{verbatim}
APP_TOKEN=xapp-...
BOT_TOKEN=xoxb-...
SIGNING_SECRET=...
SHEET=   .   .   .
TABLE=   .   .   .
ADMINS=ID ID ID ID
\end{verbatim}

The values can be found in the following ways:
\begin{itemize}
	\item \texttt{APP\_TOKEN}: Slack apps dashboard $\to$ Basic Information $\to$ App-Level Tokens $\to$ Token
	\item \texttt{BOT\_TOKEN}: Slack apps dashboard $\to$ OAuth \& Permissions $\to$ Bot User OAuth Token
	\item \texttt{SIGNING\_SECRET}: Slack apps dashboard $\to$ Basic Information $\to$ App Credentials $\to$ Signing Secret
	\item \texttt{SHEET}: Google Sheets spreadsheet $\to$ Section in URL between \texttt{spreadsheets/d/} and \texttt{/edit}
	\item \texttt{TABLE}: Google Sheets spreadsheet $\to$ Text on the highlighted tab along the bottom of the website
	\item \texttt{ADMINS}: Space-separated list of the Slack usernames of people who have permission to administrate
\end{itemize}

\subsection{The \texttt{credentials.json} File and Google Authentication}

\

Go through the ``Set up your environment" section of \href{https://developers.google.com/sheets/api/quickstart/python#set-up-environment}{Google's Python quickstart document}. Put the resultant \texttt{credentials.json} file in the same directory as the Python files. When you first run the program (or whenever you lack a \texttt{manifest.json}), you will be prompted by Google's authentication flow. Open the stated URL and log in with a Google account that has access to the spreadsheet.

\section{Administration}

/

Only Slack accounts with their usernames listed in \texttt{ADMINS} may use administration commands.

\subsection{The \texttt{Meeting} Modal}

\

For user input, the \texttt{Meeting} modal is identical to the \texttt{Attend} modal. If the date isn't in the first column, the bot will add it. Otherwise, it will select the row that the date is in. Finally, it will add the hours to the second column (Total Meeting Hours).

\subsection{The \texttt{/set sheet} and \texttt{/set table} Commands}

\

Take one argument after the sub-command (``sheet" or ``table"). Change the \texttt{SHEET} and \texttt{TABLE} values of the settings to that argument.

\subsection{The \texttt{/set op} and \texttt{/set deop} Commands}

\

Take one slack username as an argument after the sub-command (``op" or ``deop"). Add someone to or remove someone from the list in the \texttt{ADMINS} setting value.

\end{document}
