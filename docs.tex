\documentclass{article}
\usepackage[english]{babel}
\usepackage{amsmath}
\newcommand{\tmtextmd}[1]{\text{{\mdseries{#1}}}}

\begin{document}

\title{Attendbot 9000 Usage and Administration}

\author{Saul Statman}

\maketitle

\section{Usage}

\subsection{Input}

\

Input is given to the program via sending direct messages to a Slack bot account. Further details in the ``Administration" section.

\subsection{Date and Time Format}

\

The format used is \texttt{[MM/DD/YYYY] HH[:MM][am/pm] HH[:MM][pm/am]}. In the following section, a ``word'' is part of the format separated by strings. Any content enclosed in brackets is optional - it can be omitted and the program will fall back to a default value.

The first word is \texttt{[MM/DD/YYYY]}, which is a date formatted as MONTH-DAY-YEAR with forward slash separators. When it is omitted, it falls back to the current date.

The second and third words are bot \texttt{HH[:MM][am/pm]}. The only mandatory portion is the hour portion, which is a number from ``1'' to ``12''. The next section is the colon and a number from ``00'' to ``59'' representing minutes, defaulting to ``00'' when omitted. The final section is a literal ``am'' or ``pm'', representing the ante meridiem or post meridiem clarification needed for 12-hour time. The final section defaults to ``am'' in the first word and ``pm'' in the second.

\section{Administration}

\

\end{document}
