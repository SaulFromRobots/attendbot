\documentclass[english]{article}
\usepackage[latin9]{inputenc}
\usepackage{babel}
\usepackage{rotating}
\usepackage[unicode=true, bookmarks=false, breaklinks=false,pdfborder={0 0 1},backref=section,colorlinks=false] {hyperref}
\makeatletter
\providecommand{\tabularnewline}{\\}
\usepackage[english]{babel}
\makeatother

\begin{document}

\title{Attendbot 9000 Usage and Administration}
\author{Saul Statman}
\maketitle

\section{Usage}

\subsection{Input}

Input is given to the program via running shortcuts provided by a
Slack bot account. These shortcuts may be accessed from the ``{[}/{]}''
menu on the message composition bar or by using the search bar. The
bot also makes use of Slack usernames, which are the identifier used
for ``@'' tagging (excepting the ``@'').

\subsection{The \texttt{Attend} Modal}

\texttt{Attend} is the primary mode of interaction for users. The
shortcut will show a modal which takes input for the date, start time,
and end time. The program will read these arguments and edit the table
provided. To do so, it takes in your Slack username, the date in your
message, and the amount of hours between the begin and end times in
your message (subtracting one hour if you were present for the lunch
hour, using decimals for partial hours). The program will then add
the amount of hours at the cell indexed by your username and the date.

\subsection{Screens}

The bot's user page includes the ``Home'', ``Messages'', and ``About''
screens.
\begin{itemize}
\item The home screen contains the attendance data of the user currently
viewing it. This includes regular meeting and outreach numbers, as
well as a section which calculates their eligibility for both the
build team and travel team.
\item The messages screen is used for responding to user action with status
notifications.
\item The about screen has usage documentation for the various modals.
\end{itemize}

\section{Deployment}

\subsection{Creating the Bot}

First, you must go to \href{https://api.slack.com/apps}{Slack's apps panel}and
log in. Create a new app and go to the App Manifest page. Paste in
the JSON text stored in this repository's \texttt{manifest.json} file.

\subsection{The Spreadsheet}

The bot is designed to interface with Google Sheets spreadsheets in
the following format. Note the names being on column C onward, and
the dates being row 4 onward.
\begin{center}
\begin{tabular}{|c||c|c|c|c}
\hline 
 & \begin{turn}{90}
Total Meeting Hours
\end{turn} & \begin{turn}{90}
Slack username
\end{turn} & \begin{turn}{90}
Slack username
\end{turn} & . . . \tabularnewline
\hline 
Member Total &  &  &  & . . .\tabularnewline
\hline 
\hline 
Extra &  &  &  & . . .\tabularnewline
\hline 
MM/DD/YYYY &  &  &  & . . .\tabularnewline
\hline 
. . . & . . . & . . . & . . . & . . .\tabularnewline
\end{tabular}
\par\end{center}

This can be modified by changing every instance of ``4'' and ``C''
in the code's usage of table ranges. Non-date values can be used for
the date column, but to integrate with the bot they should include
the date at the beginning and separate it from all other cell contents
with a space character (eg, ``8/27/2024 (Robonanza)''). The ``extra''
row may be used for anything.

\subsection{The \texttt{keys} File}

The \texttt{keys} file contains a set of keys and values in the below
format, requiring the below keys. Values are omitted, except for the
leading differentiator of the Slack tokens. 
\begin{verbatim}
APP_TOKEN=xapp-...
BOT_TOKEN=xoxb-...
SIGNING_SECRET=
SHEET=
MEETING_TABLE=
OUTREACH_TABLE=
MEETING_REQ=
OUTREACH_REQ=
ADMINS=
\end{verbatim}
The values can be found in the following ways: 
\begin{itemize}
\item \texttt{APP\_TOKEN}: Slack apps dashboard $\to$ Basic Information
$\to$ App-Level Tokens $\to$ Token 
\item \texttt{BOT\_TOKEN}: Slack apps dashboard $\to$ OAuth \& Permissions
$\to$ Bot User OAuth Token 
\item \texttt{SIGNING\_SECRET}: Slack apps dashboard $\to$ Basic Information
$\to$ App Credentials $\to$ Signing Secret 
\item \texttt{SHEET}: Google Sheets spreadsheet $\to$ Section in URL between
\texttt{spreadsheets/d/} and \texttt{/edit} 
\item \texttt{MEETING\_TABLE}, OUTREACH\_TABLE: Google Sheets spreadsheet
$\to$ Text on the highlighted tab along the bottom of the website 
\item \texttt{MEETING\_REQ}, \texttt{OUTREACH\_REQ}: The numbers that user
attendance total should be compared against. If the table's ``member
total'' row measures attendance as a percentage, then the percentage
(number between 0 and 100) should be used, omitting the ``\%'' symbol.
\item \texttt{ADMINS}: Space-separated list of the Slack usernames of people
who have permission to administrate.
\end{itemize}
The file is only read when starting the bot, so any alterations made
while the bot is running will be lost. One should instead use the
graphical settings menu described in section 3.2 (``Settings'').

\subsection{The \texttt{credentials.json} File and Google Authentication}

Go through the ``Set up your environment'' section of \href{https://developers.google.com/sheets/api/quickstart/python\#set-up-environment}{Google's Python quickstart document}.
Put the resultant \texttt{credentials.json} file in the same directory
as the Python files. When you first run the program (or whenever you
lack a \texttt{manifest.json}), you will be prompted by Google's authentication
flow. Open the stated URL and log in with a Google account that has
access to the spreadsheet.

\section{Administration}

Only Slack accounts with their usernames listed in \texttt{ADMINS}
may use the meeting modal and settings.

\subsection{The \texttt{Meeting} Modal}

For user input, the \texttt{Meeting} modal is identical to the \texttt{Attend}
modal. If the date isn't in the first column, the bot will add it.
Otherwise, it will select the row that the date is in. Finally, it
will add the hours to the second column (``Total Meeting Hours'').

\subsection{Settings}

The settings section of the home screen is only shown to admins. It
includes text fields which may be used to alter the values of the
\texttt{SHEET}, \texttt{MEETING\_TABLE}, \texttt{OUTREACH\_TABLE},
\texttt{MEETING\_REQ}, \texttt{OUTREACH\_REQ}, and \texttt{ADMINS}
keys in the \texttt{keys} file. These settings are also changed in
the running instance of the bot.
\end{document}
