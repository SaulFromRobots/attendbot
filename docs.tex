\documentclass{article}
\usepackage[english]{babel}
\usepackage{amsmath}

%%%%%%%%%% Start TeXmacs macros
\newcommand{\tmtextmd}[1]{\text{{\mdseries{#1}}}}
%%%%%%%%%% End TeXmacs macros

\begin{document}

\title{Attendbot 9000 Usage and Administration}

\author{Saul Statman}

\maketitle

\section{Usage}

\subsection{Date and Time Format}

\

Input is given to the program via sending direct messages to a Slack bot
account.

\subsection{Input}

\

The format used is  [MM/DD/YYYY] HH[:MM][am/pm] HH[:MM][pm/am] . In the
following section, a ``word'' is a part of the format separated by strings.
Any content enclosed in brackets is optional - it can be omitted and the
program will fall back to a default value.

The first word is  [MM/DD/YYYY], which is a date formatted as MONTH-DAY-YEAR
with forward slash separators. When it is omitted, it falls back to the
current date.

The second and third words are both  HH[:MM][am/pm] . The only mandatory
portion is the hour portion, which is a number from ``1'' to ``12''. The next
section is the colon and a number from ``00'' to ``59'' representing minutes,
defaulting to ``00'' when omitted. The final section is a literal ``am'' or
``pm'', representing the ante meridiem or post meridiem clarification needed
for 12-hour time. The final section defaults to ``am'' in the first word and
``pm'' in the second.

\section{Administration\tmtextmd{}}

\

\end{document}
